\documentclass[a4paper,10pt]{article}
\usepackage[margin=1.0in]{geometry}
\linespread{1.5}

%opening
\title{\textbf{Automatic Content Based Image Organization and Ranking on Mobile devices}}

\author{Anup Mohan\\
ECE, Purdue University\\
mohan11@purdue.edu\\
\and
Youngsol Koh\\
ECE, Purdue University\\
koh0@purdue.edu\\
\and
Richa Agarwal\\
ECE, Purdue University\\
agarwa82@purdue.edu\\
}


\begin{document}

\maketitle



\begin{abstract}
A number of mobile phone and tablet users have increased rapidly over the past decade. The mobile devices are equipped with 
high quality cameras for the purpose of taking images and videos. It is common to have thousands of images in a mobile device,
and it is cumbersome to manually group the images based on its content and decide whether an image is of good qualtiy or not.
Hence it is important to automatically organize and rank the images based on its content. In this paper we develop a mobile 
application to organize images based on, (i) the facial feature information of the users, (ii) computer generated images \textit{meme} versus images 
taken from a camera. Our application will also rank images based on a no-reference blur metric and decide whether the image is 
of good quality or not. We evaluate the feasibility of executing the state of the art no-reference blur estimation methods on a 
mobile device in real time. We also evaluate the performance of the no-reference blur estimation methods on the state of the 
art image resolutions. We intend to develop a fast no-reference blur estimation method using integral images and by varying the
image resolution for real time blur estimation on mobile devices.
\end{abstract}


\end{document}